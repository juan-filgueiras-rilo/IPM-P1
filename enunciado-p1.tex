\documentclass[11pt,a4paper]{article}

\usepackage[spanish]{babel}
\usepackage[utf8]{inputenc}
\usepackage{url}


\title{Práctica 1 - Interfaces Gráficas de Usuaria con Gtk+ y Python}
\author{Interfaces Persona Máquina}
\date{18/9/2017 -- 13/10/2018}



\begin{document}


\maketitle

\renewcommand{\abstractname}{Objetivos}
\begin{abstract}
  Aplicar los conocimientos adquiridos sobre el desarrollo de
  interfaces gráficas de usuaria durante el desarrollo de una
  aplicación gráfica de escritorio. Dicha aplicación se implementará
  utilizando el lenguaje de programación Python y la librería gráfica
  Gtk+.
\end{abstract}


%%%%%%%%%%%%%%%%%%%%%%%%%%%%%%%%%%%%%%%%%%%%%%%%%%%%%%%%%%%%%%%%%%%%%%%%%%%
\section{Descripción del trabajo}

Al llegar a tu nuevo trabajo te encuentras una aplicación desarrollada
por el equipo que acaban de despedir. La aplicación es el mítico
gestor de una lista de tareas donde puedes añadir tareas pendientes de
realizar, indicar su fecha de finalización, marcar las que ya se han
realizado, etc.

Tu jefa te ha encargado que rehagas la aplicación porque se ha dado
cuenta que presenta muchos fallos. El diseño de la interface es muy
pobre y presenta muchas carencias, y el diseño software es,
simplemente, inexistente.

Como tu jefa confía en ti y sabe que vas a realizar un buen trabajo,
también te ha encargado que después de rehacer la aplicación, le
añadas las funcionalidades que están pendientes.

%%%%%%%%%%%%%%%%%%%%%%%%%%%%%%%%%%%%%%%%%%%%%%%%%%%%%%%%%%%%%%%%%%%%%%%%%%%
\section{Cómo realizar la práctica}
\label{sec:como-realizar}

En los siguientes apartados se describen las tareas que debes llevar a
cabo. Si deseas que tu trabajo reciba una valoración positiva es
necesario que te ciñas a las siguientes indicaciones:

\begin{itemize}
\item Las tareas deben realizarse en el orden en que se presentan en
  el enunciado de la práctica.

  Por supuesto, es posible un desarrollo iterativo en el que se
  corrigen errores que no se hayan detectado durante la desarrollo de
  tareas anteriores.

  Es cierto que el orden propuesto no encaja con una metodología
  actual de ingeniería del software, sin embargo, no debemos olvidar
  el objetivo de la práctica, cada tarea esta pensada para reforzar el
  aprendizaje de aspectos concretos de la materia y, para que este
  aprendizaje sea lo más efectivo posible, deben realizarse en el
  orden estipulado.

\item No se evaluará ningún material que no esté en el
  repositorio. Esto incluye el código fuente, pero también la
  documentación, diagramas, datos, ... Por tanto, no olvides incluir
  en el repositorio todos los resultados de cada una de las tareas.

  Tampoco olvides que estás trabajando con un sistema de control de
  versiones distribuido. Trabaja lo más posible contra tu repositorio
  local, y usa los repositorios para sincronizar tu trabajo con el de
  tu compañero de prácticas. Y no tengas miedo de hacer commits, no
  racanees, el sistema se encarga de calcular y almacenar sólo las
  diferencias entre una versión y la siguiente para optimizar el
  espacio.

\item Las ramas de versiones del repositorio se usarán para tomar
  decisiones a la hora de evaluar la práctica.  A continuación te
  mostramos unos ejemplos en los que la práctica se considera no apta:

  \begin{itemize}
  \item El orden de las versiones no es el indicado para las tareas en
    el enunciado de la práctica.
  \item Todas las versiones (commits) se realizaron el último día.
  \end{itemize} 
  
\item El repositorio de referencia para la evaluación de la práctica
  es el que se encuentra en los servidores proporcionados por el
  Cecafi. Asegúrate de que su contenido este completo.

  Recuerda que en cualquier momento puedes clonar el repositorio para
  comprobar su contenido. Por ejemplo, una vez clonado \texttt{git
    reset --hard sprint1} nos ``resetea'' nuestra copia de trabajo
  hasta la versión que habíamos etiquetado como \texttt{sprint1}.

\item EL proyecto que contiene el repositorio en el servidor del
  Cecafi se tiene que llamar \emph{exactamente}: \texttt{ipm-p1}.
\end{itemize}


\subsection{Requisitos no funcionales}
\begin{itemize}
\item La implementación se realizará con python y GTK+. Ambas en su
  versión 3.
\end{itemize}

%%%%%%%%%%%%%%%%%%%%%%%%%%%%%%%%%%%%%%%%%%%%%%%%%%%%%%%%%%%%%%%%%%%%%%%%%%%
\section{Descripción de la aplicación}

La aplicación es la típica \textit{TODO List} que gestiona una lista
de tareas pendientes y realizadas. Cada tarea viene dada por una
descripción textual, una fecha límite y una variable que indica si se
ha realizado o no. Los casos de uso soportados son los siguientes:

\begin{itemize}
 \item Añadir una tarea a la lista de tareas
 \item Eliminar una tarea de la lista de tareas
 \item Editar los datos de una tarea
 \item Marcar/desmarcar una tarea como hecha
 \item Ordenar las tareas 
\end{itemize}

%%%%%%%%%%%%%%%%%%%%%%%%%%%%%%%%%%%%%%%%%%%%%%%%%%%%%%%%%%%%%%%%%%%%%%%%%%%
\section{Tareas a realizar}
Los siguientes apartados describen las \emph{tareas} que debes
realizar según la planificación establecida.


%%%%%%%%%%%%%%%%%%%%%%%%%%%%%%%%%%%%%%%%%%%%%%%%%%%%%%%
\subsection{Reingeniería. Documentar}
\label{subsec:documentar}

A los defectos de la interface actual se añade el hecho de que no está
documentada, así que comenzaremos por corregir esta carencia.

\begin{enumerate}
\item Documenta la interface actual. Documenta las ventanas y
  diálogos existentes mediante \textit{wireframes}, y el flujo de
  todas las acciones posibles.

  Se exhaustivo, no te limites al comportamiento esperado de la
  interface, documenta también las situaciones de error aunque no se
  traten en la implementación inicial. Por ejemplo: pulso el botón
  'Eliminar' sin haber seleccionado una tarea.
\end{enumerate}

A la última versión que se corresponda con la realización de este
apartado asígnale la etiqueta \texttt{task1}.

%%%%%%%%%%%%%%%%%%%%%%%%%%%%%%%%%%%%%%%%%%%%%%%%%%%%%%%
\subsection{Reingeniería. Diseño software}

Dado que el diseño software es inexistente, pasamos a corregirlo en
este apartado para facilitar la realización de las siguientes tareas.

\begin{enumerate}
\item Estudia el código existente. Difícilmente podrás hacer ningún
  cambio si no lo entiendes.

\item Realiza un diseño software de la aplicación aplicando el patrón
  \emph{Model View Controller}.

  A estas alturas de la carrera no hace falta que te lo diga, pero ya
  tú sabes, usa diagramas UML para documentar el diseño.

\item Reimplementa la aplicación siguiendo el diseño que acabas de realizar.
\end{enumerate}

A la última versión que se corresponda con la realización de este
apartado asígnale la etiqueta \texttt{task2}.

%%%%%%%%%%%%%%%%%%%%%%%%%%%%%%%%%%%%%%%%%%%%%%%%%%%%%%%
\subsection{Reingeniería. Diseño e implementación}

Llegados a este punto, procede a subsanar los fallos de la
interface. Corrige el diseño de la misma e implementa los cambios.

Los principales aspectos a tratar son los siguientes:

\begin{itemize}
\item Gestionar los errores. Recuerda que ya te fijaste en ellos en la
  tarea del apartado~\ref{subsec:documentar}.

\item Mejora la eficiencia del uso de la interface. En la mayor parte
  de los casos, esto implicará reducir el número de pasos necesarios
  para realizar un caso de uso. Recuerda que tienes diagramas de
  flujo en la documentación de la interface donde este aspecto se ve
  claramente.

\item Dado que la interface se implementa con la librería Gtk+, su
  entorno de ejecución natural es el escritorio Gnome. Por tanto,
  debes adecuar el diseño de la interface a las Gnome
  HIG\footnote{\url{https://developer.gnome.org/hig/stable/}}.

  Crea un documento de texto con los apartados de cada una de las
  guías que aplica a la interface de la aplicación. En los casos en
  que sea necesario cambia el diseño para que se ajuste a las guías.
\end{itemize}

Durante la realización de este apartado, no te sorprendas por el hecho
de que la intersección de los items propuestos no está vacía. Dicho de
otra forma, para cumplir alguna de las guías tendrás que mejorar
algunos aspectos del uso de la interface o gestionar mejor los
errores.

Respecto al control de versiones, dentro de este apartado \emph{no
  hagas commits que incluyan más de un cambio en la interface}.

A la última versión que se corresponda con la realización de este
apartado asígnale la etiqueta \texttt{task3}.

%%%%%%%%%%%%%%%%%%%%%%%%%%%%%%%%%%%%%%%%%%%%%%%%%%%%%%%
\subsection{Nuevo caso de uso}

En este paso, debes añadir un nuevo caso de uso a la aplicación:

\begin{itemize}
\item Sincronizar la lista de tareas con un servidor remoto.
\end{itemize}

Para simplificar el desarrollo, el servidor remoto será simulado
mediante una función que recibe una lista de tareas, como efecto
colateral simula la sincronización a través de un servicio de red y
devuelve un valor indicando si la operación tuvo éxito o si se produjo
algún error.

Dada la naturaleza de las operaciones de red, debes tener especial
cuidado con las cuestiones relativas a la usabilidad como no
interrumpir el funcionamiento de la interface o proporcionar el
\textit{feedback} adecuado.


A la última versión que se corresponda con la realización de este
apartado asígnale la etiqueta \texttt{task4}.

%%%%%%%%%%%%%%%%%%%%%%%%%%%%%%%%%%%%%%%%%%%%%%%%%%%%%%%
\subsection{Internacionalización}

Finalmente, la última tarea será internacionalizar la aplicación para
que se ajuste automáticamente a la configuración de entorno local
(\textit{locale}) del usuario.

En concreto, se deben adaptar el idioma del texto de la interface y el
formato de las fechas.

A la última versión que se corresponda con la realización de este
apartado asígnale la etiqueta \texttt{task5}.


%%%%%%%%%%%%%%%%%%%%%%%%%%%%%%%%%%%%%%%%%%%%%%%%%%%%%%%%%%%%%%%%%%%%%%%%%%%%
\section{Evaluación de la práctica}

Una vez terminada la práctica tienes que hacer una presentación de la
misma al profesor para que la evalúe.

Para evaluar la práctica, el profesor seguirá los siguientes
criterios:

\begin{itemize}
\item Si el repositorio no cumple con lo establecido en la
  sección~\ref{sec:como-realizar}, la práctica se evaluará como
  \emph{no apta}, nota numérica 0.

\item Si alguno de los integrantes del grupo, por su desconocimiento
  de la práctica, muestra no haber participado en la realización de la
  misma, esa persona recibirá una evaluación en la práctica de
  \emph{no apta}.

\item Cada una de las tareas que conforma la práctica se evaluarán de
  la siguiente manera:

  \begin{itemize}
  \item \texttt{task1}
    \begin{itemize}
    \item Hasta 0,75 puntos si se han documentado correctamente todas
      las ventanas y diálogos.
    \item Hasta 0,5 puntos si se han documentado correctamente todos
      los flujos de las acciones cuando no hay errores.
    \item Hasta 0,75 puntos si se han documentado correctamente todos
      los flujos de las acciones cuando hay errores.
    \end{itemize}
  
  \item \texttt{task2}

    \begin{itemize}
    \item Hasta 0,75 puntos si el diseño sigue el estándar UML.
    \item Hasta 0,75 puntos si el diseño sigue el patrón MVC.
    \item Hasta 0,5 puntos si la implementación se corresponde con el
      diseño.
    \item -0,5 puntos si el diseño no es anterior a la
      implementación\footnote{Recuerda que estos aspectos se verifican
        con las versiones del repositorio.}.
    \end{itemize}

  \item \texttt{task3}

    \begin{itemize}
    \item Hasta 1 punto si la interface gestiona los errores.
    \item Hasta 0,75 puntos si se ha mejorado la eficiencia de uso de
      la interface.
    \item Hasta 0,75 si la interface sigue las Gnome HIG.
    \end{itemize}

  \item \texttt{task4}

    \begin{itemize}
    \item Hasta 2 puntos si la sincronización no bloquea la interface.
    \item Hasta 0,5 puntos si se proporciona un feedback adecuado.
    \end{itemize}

  \item \texttt{task5}

    \begin{itemize}
    \item Hasta 0,5 puntos si los textos de la interface se adaptan
      al idioma de la usuaria.
    \item Hasta 0,5 puntos si las fechas se adaptan al entorno local
      de la usuaria.
    \end{itemize}
  \end{itemize}

\item El resultado de evaluar las tareas de la práctica constituye la
  nota provisional de la misma. A continuación, si procede, se aplican
  las penalizaciones por finalizar la práctica fuera de plazo.

  \begin{itemize}
  \item Si la práctica se finalizó dentro del plazo establecido, esta
    nota se convierte en la nota final.
  \item Si la práctica se finalizó fuera del plazo establecido:
    \begin{itemize}
    \item Cuando finaliza con menos de un semana de retraso, la nota
      final es el mínimo entre 8,99 y la nota provisional.
    \item Cuando finaliza con más de un semana de retraso, la nota
      final es el mínimo entre 6,99 y la nota provisional.
    \end{itemize}
  \end{itemize}

\end{itemize}


\end{document}
